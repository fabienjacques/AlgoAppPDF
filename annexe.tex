\begin{appendix}
\section{Définitions}
\subsection{Ensemble dominant}
\label{def:ensemble_dominant}
Un ensemble dominant d'un graphe $G(V, E)$ est un sous ensemble $D \subseteq V$ de sommets tel que tout sommet de $G$ est soit dans $D$ soit adjacent à un sommet de $D$.
\subsection{Nombre de domination}
\label{def:nombre_de_domination}
Le nombre de domination $\gamma(G)$ d'un graphe $G$ est le cardinal d'un plus petit ensemble dominant de $G$.
\subsection{Ensemble de sommets stable}
\label{def:ensemble_de_sommets_stable}
Soit $G(V, E)$ un graphe, un sous-ensemble $S \subseteq V$ de sommets est stable si aucune paire de sommets dans $S$ n'est reliée par une arête de $G$.
\end{appendix}